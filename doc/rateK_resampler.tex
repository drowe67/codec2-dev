\documentclass{article}
\usepackage{amsmath}
\usepackage{xstring}
\usepackage{catchfile}
\usepackage{graphicx}
\graphicspath{ {../octave/} }
\usepackage{url}

\CatchFileDef{\headfull}{../.git/HEAD}{}
\StrGobbleRight{\headfull}{1}[\head]
\StrBehind[2]{\head}{/}[\branch]
\IfFileExists{../.git/refs/heads/\branch}{%
    \CatchFileDef{\commit}{../.git/refs/heads/\branch}{}}{%
    \newcommand{\commit}{\dots~(in \emph{packed-refs})}}
\newcommand{\gitrevision}{%
  \StrLeft{\commit}{7}%
}

\title{Codec 2 Rate K Resampler}
\author{David Rowe\\ \\ Revision: {\gitrevision} on branch: {\branch}}
\date{\today}
\begin{document}

\maketitle

\section{Introduction}

The Rate K resampler is used in Codec 2 700C to transform the variable length vectors of spectral amplitude samples to fixed length $K$ vectors suitable for vector quantisation.  This document was written in order to explore and possibly improve the resampler.

\section{Theoretical Model}

Consider a vector $\mathbf{a}$ of $L$ spectral amplitudes, sampled at time $t=nT$ seconds, where $n$ is the frame number, and $T$ is the frame period, typically $T=0.01$ seconds. 
\begin{equation}
\mathbf{a} = \begin{bmatrix} A_1, A_2, \ldots A_L \end{bmatrix} 
\end{equation}
$A_m$ is sampled at the frequency $f_m=mF0$ Hz for $m=1 \ldots L$, where $F0$ is the fundamental frequency (pitch) in Hz of the current frame, and $L$ is given by:

\begin{equation}
L=\left \lfloor \frac{F_s}{2F0} \right \rfloor
\end{equation}
$F0$ and $L$ are time varying as the pitch track evolves over time. For speech sampled at $F_s=8$ kHz $F0$ is typically in the range of 50 to 400 Hz, giving $L$ in the range of 10 $\ldots$ 80. \\

To quantise and transmit $\mathbf{a}$, it is convenient to resample $\mathbf{a}$ to a fixed length $K$ element vector $\mathbf{b}$ using a resampling function:
\begin{equation}
\mathbf{b} = \begin{bmatrix} B_1, B_2, \ldots B_K \end{bmatrix} = R(\mathbf{a})
\end{equation}
To model the logarithmic frequency response of the human ear $B_k$  are sampled on non-linearly spaced points on the frequency axis $f_k=warp(k)$ Hz for $k=1 \ldots K$, where $warp(k)$ is a frequency warping function. A typical value of $K$ is 20. A typical choice for $warp()$ is the Mel function which samples the spectrum more densely at low frequencies, and less densely at high frequencies.  

The rate $K$ vector $\mathbf{b}$ is vector quantised for transmission over the channel:
\begin{equation}
\hat{\mathbf{b}} = Q(\mathbf{b})
\end{equation}
The rate $L$ vector can then be recovered by resampling $\mathbf{\hat{b}}$ using another resampling function:
\begin{equation}
\hat{\mathbf{a}} = S(\hat{\mathbf{b}})
\end{equation}
A useful error metric is the mean square error:
\begin{equation}
\begin{split}
E & =\frac{1}{L_{max}-L_{min}+1}\sum_{m=L_{min}}^{L_{max}}(A_m-\hat{A}_m)^2 \\
L_{min} & = round(200/F0) \\
L_{max} & =\left \lfloor 3700/F0  \right \rfloor
\end{split}
\end{equation}
If $A_m$ are in dB, $E$ can be denoted the spectral distortion in $dB^2$, which can be averaged over a testing database of $N$ frames to obtain mean spectral distortion.

Consider a choice of $warp()$ with linear (non-warped) sampling of the frequency axis, and an ideal quantiser $Q$ such that $\hat{\mathbf{b}} = \mathbf{b}$. If $K<L$ information may be lost due to undersampling, which implies $\hat{\mathbf{a}} \neq \mathbf{a}$.  
With nonlinear sampling, there will be local undersampling where the sampling rate of $\mathbf{b}$ is less than that of $\mathbf{a}$:
\begin{equation}
warp(k+1)-warp(k) < F0
\end{equation}
If $A_m$ is changing rapidly, undersampling may introduce undesirable aliasing, which may manifest as noise that is superimposed on $\mathbf{b}$. This noise may reduce perceptual quality and consume valuable quantiser bits for no benefit. Given the ear is less sensitive to detail at high frequencies, it is reasonable to choose a resampling function that smooths high frequency detail such that local undersampling and uncontrolled aliasing is minimised. A useful property of $R$ may therefore be smoothing (filtering) $\mathbf{a}$ such that $E$ is small when $\hat{\mathbf{b}} = \mathbf{b}$.  The filter and number of sample points $K$ should also be chosen to minimise the perceptual distortion.  A suitable filtering function would average $A_m$ over the region $warp(k+1)-warp(k)$.

\section{Suggested Experiments}
 
Here are a suggested set of experiments to evaluate the ideas presented in this document.  Some of them require informal listening tests, others have objective measures which could be used as the basis for automated tests:

\begin{enumerate}

\item The baseline resampling (currently a 2nd order polynomial) is a potential source of distortion.  Conduct an experiment to test the theory that $E$ is small (and perceptual quality high) for $K>L$ using linear frequency sampling and large $K$.

\item How to demonstrate aliasing?  Well we can run current rate K code, that uses a 2nd order parabolic resampler.  Then compare with a "better" resampler that uses filtering.  Perform an informal listening test over a small set of samples.  Goal is to show reduced $E$ with similar perceptual quality.  Smoothing does reduce information so there will be a trade off.  Too much smoothing and perceptual quality will reduce.  We should also notice improved VQ performance, as we won't be quantising noise.

\item A useful property is sensitivity to quantisation, which could be defined as $\frac{\partial E}{\partial \mathbf{b}}$. For example, given a 1dB RMS error in the elements of $\mathbf{b}$, what is the impact on $E$?

\item To minimise bit rate, it is common to transmit $\mathbf{b}$ to the receiver at period $T/D$ seconds, where $D$ is the decimation ratio, and discarding the intermediate $D-1$ frames. A useful property is the ability to smoothly interpolate between transmitted frames $\mathbf{b}_n$ and $\mathbf{b}_{n+D}$ to recover $\mathbf{b}_{n+i}$ where $i=1 \ldots D-1$.  Need a definition for smoothness.

\item Speech evolves slowly over time compared to the $T=0.01$ second frame period.  Adjacent frames of speech parameters such as $\mathbf{a}_n$ and $\mathbf{a}_{n+1}$ have some correlation which can be used to obtain coding efficiency:
\begin{equation} \label{eq_delta}
\begin{split}
\mathbf{b}_{n+1} & = \mathbf{b}_{n} + \mathbf{\Delta}_n \\
\mathbf{\Delta}_{n+1} & = \mathbf{b}_{n+1} - \mathbf{b}_{n}
\end{split}
\end{equation}

In general $\mathbf{\Delta}_n$ can be encoded with less bits than $\mathbf{b}_n$.  However consider the case where there is significant noise due to undersampling:
\begin{equation}
\mathbf{\hat{b}}_n = \mathbf{b}_{n} + \mathbf{n}_{n}
\end{equation}
where $\mathbf{n}_{n}$ is a vector of noise samples with an unknown distribution. Substituting into \ref{eq_delta}:
\begin{equation}
\mathbf{\Delta}_{n+1} = \mathbf{b}_{n+1} - \mathbf{b}_{n} + \mathbf{n}_{n+1} - \mathbf{n}_{n}
\end{equation}
If $\mathbf{n}_{n}$ and $\mathbf{n}_{n+1}$ are not well correlated they may become a significant source of noise that is summed with $\mathbf{\Delta}_{n}$, reducing the effectiveness of the quantister that will need to waste bits quantising the noise. We would therefore expect that in the absence of undersampling noise, delta coding in time should result in increased quantiser efficiency.

\item Small changes in $\mathbf{a}$ input should result in small changes in $\mathbf{b}$ indicating a lack of sensitivity and undersampling noise in $R$.  If $R$ is sensitive, we may notice VQ choices changing from frame to frame for stationary speech.

\item If the sample rate $K$ is sufficiently high (or bandwidth of $\mathbf{a}$ sufficiently constrained), the actual VQ dimension won't matter.  The decorrelation properties of the VQ will ensure it achieves the same distortion over a range of dimensions.  A large enough dimension $K$ could be chosen to simplify $S$, which could be linear resampling. It would be good to decouple $warp()$ from K.
\end{enumerate}

\section{Experiment 1: rate $K>L$ linear}

This experiment tests the ability to resample perfectly ($E=0$) with $K>L$, and linear frequency sampling and was implemented with Octave scripts \path{ratek1_fbf.m} and \path{ratek1_batch.m}.

Figure \ref{fig:ratek1_big_dog_50} illustrates why the resampler choice is important.  In the region of F1, the spectral samples $A_{m}$ are changing quickly.  The parabolic resampler fails to track these changes leading to distortion in a perceptually important feature of the spectrum.  When the resampler is viewed as a filter, this can be interpreted as a low pass response in the parabolic resampler, and is most noticable when $K$ is close to $L$.

\begin{figure}[h]
\caption{Frame 50 of big\_dog sample with $K=40$ just larger than $L=37$, 2nd order parabolic resampler.  Note distortion around F1 at 500 Hz }
\label{fig:ratek1_big_dog_50}
\includegraphics[width=10cm]{ratek1_big_dog_50}
\end{figure}

Figure \ref{fig:ratek1_hts_E} shows the spline and parabolic resampler $E$, plotted against $F0$ for a 24 second sample containing four speakers.  Table \ref{table:ratek1_mean_E} is the mean spectral distortion $E$.  The spline interpolator perfoms better than the parabolic interpolator.  $L$ is time varying, but as it approaches $K$, $E$ increases, once again showing that interpolators struggle with $K$ close to $L$.

\begin{table}[h]
\centering
\begin{tabular}{c c }
 \hline
 Resampler & mean $E$ $dB^2$ \\
 \hline
 spline & 0.02 \\ 
 para  & 0.31 \\
 \hline
\end{tabular}
\caption{Table to test captions and labels.}
\label{table:ratek1_mean_E}
\end{table}

In this experiment with $K=80$ both resamplers have low average $E$ (less than $1 dB^2$), and even the occasional high $E$ frames are unlikley to be audible.  With unvoiced or background noise the pitch estimator tends to low $F0$ values.  With this type of signal the ear is quite insensitive to spectral distortion $E$.  With a lower choice of $K$, this may not be the case, as we would start to get resampling distortion (Figure \ref{fig:ratek1_big_dog_50}), or when $K<L$, aliasing.

\begin{figure}[h]
\caption{Scatter plot of $E$ versus $F0$ for hts sample, spline and parabolic resampler}
\label{fig:ratek1_hts_E}
\includegraphics[width=10cm]{ratek1_hts_E.png}
\end{figure}

\begin{figure}[h]
\caption{Histogram of $E$ for spline and parabolic resampler}
\label{fig:ratek1_hts_hist}
\includegraphics[width=10cm]{ratek1_hts_hist.png}
\end{figure}

Conclusions:
\begin{enumerate}
\item The resampler matters, especially when $K$ is close to $L$.  The current rate $K$ Codec 2 700C system uses $K < L$ (at least at high frequencies) with a parabolic resampler and no filtering.  This may be suffering from resampling noise that affects VQ performance and speech quality. It seems prudent to use a low distortion resampler.  We do not want to add any additional distortion sources to our system.

\item We intend to use $K < L$, which will lead to aliasing distortion.  It would be wise to filter $A_{m}$ prior to resampling to remove the possibility of aliasing and resampler distortion, such that with $\mathbf{b}=\hat{\mathbf{b}}$, $\mathbf{a}$ = $\hat{\mathbf{a}}$. TODO reformulate this with filtering maths

\item Once filtered, we are free to use any choice of $K$ for $\mathbf{b}$ that can represent the filtered samples for vector quntisation, as the "bandwidth" of the sampled sequence (and presumably VQ distortion for a given number of bits) will be independant of $K$.  While a larger $K$ will use more VQ storage, it may simplify resampling at the receiver.  If $K$ is large, a simple linear resampler may suffice.

\item With non-linear frequency sampling, we may need a high local rate near F1 to accurately represent sharp formants, especially for males.  There may be other ways to encode this information, for a example a resampling function or VQ that takes into account $F0$ - "sharpening" F1 for low $F0$ speakers.

\item The sensitivity of F1 to low $K$ resampling also hints at the newamp1 postfilter function.  This experimentally derived algorithm sharpens formants after vector quantisation and has a large impact on Codec 2 700C speech quality.  Its function is not well understood.

\end{enumerate}

\end{document}
